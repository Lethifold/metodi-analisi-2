\documentclass[10pt,a4paper]{report}
\usepackage[utf8]{inputenc}
\usepackage{amsmath}
\usepackage{amsfonts}
\usepackage{amssymb}
\usepackage{graphicx}
\usepackage[left=2.00cm, right=2.00cm, top=2.00cm, bottom=2.00cm]{geometry}
\author{Fabio Baldo}
\title{Procedimenti risolutivi per esercizi per Analisi II}


%Nuovi comandi
\newcommand{\pdv}[3]{\frac{\partial^{#2} #1}{\partial #3^{#2}}}
\newcommand{\norm}[1]{ \left\lVert {#1} \right\rVert}

\begin{document}
	\section*{Integrazione}
	
		\subsection*{Integrali doppi}
			Gli Integrali doppi su un certo insieme $ \mathcal{D} $ (misurabile) permettono di trovare
			\[ \iint_{\mathcal{D}} f(x,y)dxdy \; = \; \text{Volume racchiuso tra la funzione integranda ($ f(x,y) $) e il piano $ xOy $} \]
			Nel caso in cui si abbia $ f(x,y) = 1 $ allora 
			\[ \iint_{\mathcal{D}} dxdy \; = \; \text{ Area dell'insieme $\mathcal{D}$  }\]
			I procedimenti che permettono di calcolare gli integrali doppi sono due
				\subsubsection{Per orizzontali}
				\begin{enumerate}
					\item Disegno approssimativamente il dominio d'integrazione ragionando prima su
					\item Svolgo prima l'integrale della funzione in $ dx $ immaginando che la variabile $ y $ sia una costante
					\item Poi integro il risultato della prima integrazione in $ dy $. In questo passaggio è vincolante ottenere un valore che non dipenda dalle due variabili $ dx \; dy $.
				\end{enumerate}
				\[ \iint_{\mathcal{D}} f(x,y) = \int_{y_{A}}^{y_{B}} \Big(\int_{x_{A}}^{x_{B}}f(x,y) dx\Big) dy \]
				\subsubsection{Per verticali}
				\begin{enumerate}
					\item Disegno approssimativamente il dominio d'integrazione ragionando prima su
					\item Svolgo prima l'integrale della funzione in $ dy $ immaginando che la variabile $ x $ sia una costante
					\item Poi integro il risultato della prima integrazione in $ dx $. In questo passaggio è vincolante ottenere un valore che non dipenda dalle due variabili $ dx \; dy $.
				\end{enumerate}
				\[ \iint_{\mathcal{D}} f(x,y) = \int_{x_{A}}^{x_{B}} \Big(\int_{y_{A}}^{y_{B}}f(x,y) dy\Big) dx \]
				
		\subsection*{Integrali tripli}
		Nel caso di integrali tripli, come per i doppi, l'integrazione permette di ricavare un ``volume'', tuttavia in questo caso l'integrale rappresenterà un ``volume'' non più in $ \mathbb{R}^{3} $, bensì in $ \mathbb{R}^{4} $. Dato che non è facile immaginarsi un volume in $\mathbb{R}^{4}$ è più congeniale pensare a $f(x,y,z)$ come una funzione che descrive l'andamento della massa nel solido e di voler calcolare la massa totale di quest'ultimo (cfr. \textit{Applicazioni alla fisica}).
		Questo tipo di integrali può essere svolto in due modi.
			\subsubsection{Per fili}
			Per immaginarsi l'integrazione per fili è utile pensare di trovare prima il volume di uno spaghetto verticale infinitamente sottile che è lungo esattamente quanto il contenitore dentro al quale si trova, ovvero l'insieme $\mathcal{P}$ e poi di sommare tutti gli spaghetti nel contenitore per trovare il volume del contenitore.
				\begin{enumerate}
					\item Integro la funzione in $ dz $ tra gli estremi dell'insieme $ \mathbb{P} $
					\item Integro i fili sulla superficie di base $ \mathcal{B} $ dell'insieme di integrazione.
					
					\[ \iiint_{\mathcal{P}} f(x,y,z) dxdydz \; = \; \iint_{\mathcal{B}} \Big(\int_{z_{A}}^{z_{B}} f(x,y,z)\Big) dxdy\]
					
				\end{enumerate}
			\subsubsection{Per strati}
			In questo procedimento ci si deve immaginare di trovare prima l'area di uno strato generico all'interno dell'insieme di integrazione e poi immagino di sommare tutti gli strati in modo di ottenere tutto il volume dell'insieme $ \mathbb{P} $
				\begin{enumerate}
					\item Per prima cosa è necessario trovare gli estremi massimo e minimo per cui si può trovare uno strato
					\item Calcolare l'area di un generico strato
					\item In fine è sufficiente integrare tutti gli strati tra l'altezza massima e la minima.
					
					\[ \iiint_{\mathcal{P}} f(x,y,z) dxdydz \; = \; \int_{z_{min}}^{z_{max}} \Big(\iint_{\mathcal{B}} f(x,y,z) dxdy\Big)dz \]
					
				\end{enumerate}
		
		\subsection*{Applicazioni alla fisica}
		Definendo la densità  di massa $ \mu (x,y,z) $ di $\mathcal{D}\subseteq \mathbb{R}^{3}$, allora si possono ricavare alcune grandezze fisiche.
			\subsubsection{Massa Totale}
			\[ M(\mathcal{D})= \iiint_{\mathcal{D}} \mu(x,y,z) dx dy dz \]
			
			\subsubsection{Baricentro}
			\[ x_{G}=\frac{\iiint_{\mathcal{D}} x \mu (x,y,x) dx dy dz}{ M(\mathcal{D})} \]
			\[ y_{G}=\frac{\iiint_{\mathcal{D}} y \mu (x,y,x) dx dy dz}{ M(\mathcal{D})} \]
			\[ z_{G}=\frac{\iiint_{\mathcal{D}} z \mu (x,y,x) dx dy dz}{ M(\mathcal{D})} \]
			
			\subsubsection{Momento d'inerzia}
			\[ \iiint_{\mathcal{D}} r^{2}(x,y,z) \mu(x,y,z) dx dy dz \]

		\subsection*{Cambiamenti di coordinate}
			\subsubsection{Polari piane}
			$ T: \begin{cases} 
				x=\rho cos(\theta) \\ 
				y=\rho sin(\theta)
			\end{cases} $
            \\
           	Lo jacobiano di questa trasformazione è $\mathcal{J}T=\rho $
				\subparagraph{Casi particolarmente favorevoli}
				Cerchio, corona circolare, "fetta di torta", "fetta di ciambella"
			\subsubsection{Polari sferiche}
			$ T: \begin{cases} 
				x=\rho sin(\varphi)cos(\theta) \\ 
				y=\rho sin(\varphi)sin(\theta) \\
				z=\rho cos(\varphi)
			\end{cases} $
			\\
			Lo jacobiano di questa trasformazione è $\mathcal{J}T=\rho sin(\varphi) $
			
				\subparagraph{Casi particolarmente favorevoli}
                Sfera, "spicchio"
                
			\subsubsection{Cilindriche}
			$ \begin{cases} 
			x=\rho cos(\theta) \\ 
			y=\rho sin(\theta) \\
			z=t
			\end{cases} $
			\\
			Lo jacobiano di questa trasformazione è $\mathcal{J}T=\rho $
			
				\subparagraph{Casi particolarmente favorevoli}
		

		\subsection*{Integrali \textbf{curvilinei}}
			Data una curva e una funzione scalare posso:
			\begin{enumerate}
				\item Parametrizzo la curva $ \gamma(t) $ sulla quale voglio calcolare l'integrale
				\item Calcolo la derivata $ \gamma ' (t) $della curva parametrizzata
                \item Calcolo la normale $ \norm{\gamma '(t)} $
				\item Calcolo la funzione composta $ f(\gamma (t)) $
				\item Determino gli estremi di integrazione
				\item Calcolo l'integrale secondo la formula:
			\end{enumerate}
		
		\[ \int_{a}^{b} f(\gamma (t)) \norm{\gamma '(t)} dt\]
		
		
		\subsection*{Integrali di \textbf{linea}}
		Data una curva e un campo vettoriale posso:
			\begin{enumerate}
				\item Parametrizzo la curva $ \gamma(t) $ sulla quale voglio calcolare l'integrale. 
				\item Calcolo la derivata della curva parametrizzata $ \gamma ' (t) $ 
				\item Calcolo la funzione composta $ \mathbf{F}(\gamma (t)) $
				\item Calcolo l'integrale dato dalla relazione:
				\end{enumerate}
			
		\[ \int_{a}^{b} \mathbf{F}(\gamma (t)) \cdot \gamma '(t) dt \]
		
		\subsection*{Integrali \textbf{superficiali}}
			\begin{enumerate}
				\item Parametrizzo la superficie $ \Sigma $ sulla quale voglio integrare
				\item Determino la normale alla superficie $\norm{\mathbf{N}(u,v)} $ sapendo che \\
                \[ \mathbf{N}(u,v) = det \left(\begin{array}{ccc} i & j & k\\ \pdv{\Sigma_{i} }{}{u} & \pdv{\Sigma_{j}}{}{u} & \pdv{\Sigma_{k}}{}{u} \\ \pdv{\Sigma_{i}}{}{v} & \pdv{\Sigma_{j}}{}{v} & \pdv{\Sigma_{k}}{}{v} \end{array}\right) \]
				\item Calcola la composta $f(\mathbf{\sigma} (u,v))$
				\item Calcolo l'integrale secondo la relazione:
			\end{enumerate}
		\[ \int_{\mathbf{\sigma}} fdS = \int_{K} f(\mathbf{\sigma} (u,v)) \norm{\mathbf{N}(u,v)}dudv \]
		
		\subsection*{Integrale di \textbf{flusso}}
		
		\begin{enumerate}
			\item Parametrizzo la superficie $ \mathbf{\sigma}(u,v) $ sulla quale devo risolvere l'integrale 
			\item Determino il vettore normale alla superficie $ \mathbf{N}(u,v) $ e calcolo anche il versore $ \mathbf{n}(u,v) $
			\item Calcolo la norma del vettore normale 
			\item Calcolo la composta $ \mathbf{F}(\mathbf{\sigma}(u,v)) $
		\end{enumerate}
		
		\[  \int_{\mathbf{\sigma}} \mathbf{F}(\vec{x}) \cdot dS = \int_{K} \mathbf{F}(\mathbf{\sigma}(u,v)) \cdot \mathbf{N}(u,v) dudv\]
		
		\subsection*{Teoremi relativi all'integrazione}
		
			\subsubsection{Guldino 1}
			Il volume di un solido di rotazione D è uguale all'area della sezione meridiana S moltiplicata per la lunghezza della circonferenza descritta dal baricentro di S attorno all'asse di rotazione.
			
			\subparagraph{Condizioni}
				\begin{itemize}
					\item Solido di rotazione 	 
				\end{itemize}
			Nel caso di una rotazione intorno all'asse z	
			\subparagraph{Teorema}		
			\[ Volume(D) = \iiint_{\mathcal{P}}dxdydz = 2\pi y_{G} A(S) \]
			
			\subsubsection{Guldino 2}
			La superficie di un solido di rotazione uguale alla lunghezza dell'arco generatore moltiplicato per la circonferenza che descrive il baricentro intorno all'asse di rotazione.
			
			\subparagraph{Condizioni}
				\begin{itemize}
					\item La superficie $ \Sigma $ deve essere data da una rotazione dell'arco $ \gamma(t)  $, appartenente ad un piano delimitato da due assi cartesiani, intorno ad uno di questi due.
				\end{itemize}
			Nel caso di una rotazione di un arco $ \gamma(t) \in yOz $ intorno all'asse $ z $ ottengo	
			
			\subparagraph{Teorema}		
			\[ Area(\Sigma) = \ 2\pi y_{G} l(\gamma) \]
			
			\subsubsection{Green}
				\subparagraph{Condizioni}
					\begin{itemize}
						\item Sia $ \mathbf{F}(x,y) = (f_{1}(x,y),f_{2}(x,y)) $ un campo vettoriale di classe $C^{1} (\Omega)$ con $\Omega \subseteq \mathbb{R}^{2}$
						\item Sia A un aperto limitato contenuto in $ \Omega $
						\item La frontiera di A è il sostegno di un arco chiuso, semplice e regolare a tratti percorso in verso antiorario.
					\end{itemize}
			
			\subparagraph{Teorema}
			
			\[ \int_{\delta A} \mathbf{F}(\vec{x}) \cdot d \mathbf{P} = \iint_{A} \Big(\pdv{f_{2}}{}{x} - \pdv{f_{1}}{}{y}\Big) dxdy \]
			
			
			\subsubsection{Gauss}
				\subparagraph{Condizioni}
					\begin{itemize}
						\item $ \mathbf{F}(\vec{x}) $ è un campo vettoriale definito in un aperto $ \Omega \in \mathbb{R}^{3} $
						\item Sia $ \Omega_{0} $ un aperto limitato la cui frontiera è $ \delta \Omega_{0} $
					\end{itemize}
				\subparagraph{Teorema}
				\[ \int_{\delta \Omega_{0}} \mathbf{F}(\vec{x}) \cdot \mathbf{n} dS = \int_{\Omega_{0}} div(\mathbf{F}(\vec{x})) dxdydz \]
			
			\subsubsection{Stokes}
			
				\subparagraph{Condizioni}
				\begin{itemize}
					\item Dato $\mathbf{F}(\vec{x})$ definito in un $ \Omega \in \mathbb{R}^{3} $
					\item Sia K il compatto costituito dal sostegno di un arco chiuso, semplice e regolare a tratti $ \gamma $ e dal suo interno.
					\item $ \gamma $ è orientata in modo da lasciare alla sinistra il suo interno
					\item Sia $ \sigma_{0} = \sigma(K) $ la calotta relativa a K e $ \delta \sigma_{0} $ l'arco detto bordo della calotta $ \sigma_{0} $  il cui verso di percorrenza è dato dalla regola della mano destra  rispetto alla normale alla calotta
				\end{itemize}
				\subparagraph{Teorema}
				\[ \int_{\delta \sigma_{0}} \mathbf{F}(\vec{x})(P) \cdot d \mathbf{P} = \int_{ \sigma_{0}} rot(\mathbf{F}(\vec{x})) \cdot d S\]




	
	\section*{Campi conservativi}
    
			Un campo vettoriale $ \mathbf{F}(\vec{x}) $, definito in $ \Omega $, si dice conservativo se 
			esiste una funzione scalare $ \varphi $ tale che per ogni $ x \in \Omega $ si abbia \[ \mathbf{F}(\vec{x}) = \nabla \varphi(\mathbf{x})\] In cui $ \varphi(\mathbf{x}) $ è detta funzione potenziale
			
			\subsection*{Condizioni}
			Affinché un campo vettoriale $ \mathbf{F}(\vec{x}) $ sia conservativo è necessario che:
			\begin{itemize}
                \item $ \mathbf{F}(\vec{x}) $ deve ammettere un potenziale $ \varphi $ tale che $ \mathbf{F}(\vec{x})= \nabla \varphi$
				\item Se $ \gamma $ è un arco chiuso e regolare a tratti, allora $ \oint_{\gamma } \mathbf{F}(\vec{x}) \cdot d\mathbb{P}  =  0$
				\item Il campo $ \mathbf{F}(\vec{x}) $ sia definito su di un insieme $ \Omega $ \textit{\textbf{semplicemente connesso}}, (nel quale una qualunque $ \gamma(t) $) chiusa racchiuda un sottoinsieme totalmente contenuto in $ \Omega $ e che $ rot(\mathbf{F}(\vec{x})) = \textbf{0} $ \\ 
			\end{itemize}
			Se un campo vettoriale $ \mathbf{F}(\vec{x}) $ è conservativo, allora:
			\begin{itemize}
				\item Dato $ \gamma $ un arco chiuso e regolare a tratti, allora $ \int_{\gamma } \mathbf{F}(\vec{x}) \cdot d\mathbb{P}  =  0$
				\item $ rot(\mathbf{F}(\vec{x})) = \textbf{0} $
			\end{itemize}
		
			\subsection*{Ricerca del potenziale}
			Per trovare il potenziale di un campo conservativo si può ragionare in più modi:

                \subsubsection{Metodo delle derivate parziali}
                    Dato il campo $ \mathbf{F}(f_{1},f_{2},f_{3}) $ cerco la funzione $ \varphi(x,y,z) $ tale per cui $ \mathbf{F}(\vec{x}) = \nabla \varphi(x,y,z) $
                \begin{itemize}
                    \item Calcolo $ \varphi = \int f_{1} dx = \varphi_{1} + h(y,z)$
                    \item Calcolo $ h(y,z) = \int (f_{2} - \pdv{\varphi_{1}}{}{y})dy$
                    \item Ricavo $ \varphi_{2} = \varphi_{1} + h(x,y) $
                    \item Ottengo la funzione potenziale  dalla relazione $\varphi(x,y,z) = \int  $  
                \end{itemize} 
				
    \section*{Serie numeriche}
        
                
            \subsection*{Serie a termini positivi}

                \subsubsection{Criterio del confronto}
                Per poter dimostrare che una serie converge utilizzando il criterio del confronto è necessario trovare una serie (più facile) maggiorante della serie di partenza. Infatti il criterio permette di dimostrare la convergenza della serie di partenza nel caso in cui la serie maggiorante risulti essere convergente.
                \subsubsection{Criterio del confronto asintotico}
                Questo criterio permette di determinare la convergenza oppure divergenza di una serie nel caso in cui si trovi una serie con stesso ordine di grandezza della serie di partenza. In questo caso è intuitivo il fatto che entrambe le serie in analisi si comportino allo stesso modo.
                \subsubsection{Criterio del rapporto}
                Per poter determinare una convergenza oppure divergenza di una serie mediante questo criterio è necessario risolvere il limite seguente \[ \lim_{n \to \infty} \frac{a_{n+1}}{a_{n}} =l \] Nel caso in cui $ l < 1 $ la serie converge, mentre nel caso in cui $ l > 1 $ la serie diverge.
                \subsubsection{Criterio della radice}
                Questo criterio è molto simile al criterio del rapporto. Infatti è necessario anche in questo caso risolvere un limite \[ \lim_{n \to \infty} \sqrt[n]{a_{n}} = l \] Nel caso in cui $ l < 1 $ la serie converge, mentre nel caso in cui $ l > 1 $ la serie diverge.
                \subsubsection{Criterio di McLaurin}
                Data una serie necessariamente positiva, continua e decrescente. Ponendo $ f(n) = a_{n} $ allora si ha che la serie $ \sum_{n=1}^{\infty} a_{n} $ e l'integrale $ \int_{1}^{\infty} f(x)dx $ hanno lo stesso comportamento 
                
            \subsection*{Serie a termini di segno variabile}
                
                \subsubsection{Teorema della convergenza assoluta}
                Se una serie è assolutamente convergente ovvero si ha la convergenza della serie $ \sum_{n=1}^{\infty} |a_{n}| $ allora la serie $ \sum_{n=1}^{\infty} a_{n} $ è anche convergente
\end{document}